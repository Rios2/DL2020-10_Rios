% Digital Logic Lab 1
% Created: 2020-01-14, Ivan Rios

%==========================================================
%=========== Document Setup  ==============================

% Formatting defined by class file
\documentclass[11pt]{article}

% ---- Document formatting ----
\usepackage[margin=1in]{geometry}	% Narrower margins
\usepackage{booktabs}				% Nice formatting of tables
\usepackage{graphicx}				% Ability to include graphics

%\setlength\parindent{0pt}	% Do not indent first line of paragraphs 
\usepackage[parfill]{parskip}		% Line space b/w paragraphs
%	parfill option prevents last line of pgrph from being fully justified

% Parskip package adds too much space around titles, fix with this
\RequirePackage{titlesec}
\titlespacing\section{0pt}{8pt plus 4pt minus 2pt}{3pt plus 2pt minus 2pt}
\titlespacing\subsection{0pt}{4pt plus 4pt minus 2pt}{-2pt plus 2pt minus 2pt}
\titlespacing\subsubsection{0pt}{2pt plus 4pt minus 2pt}{-6pt plus 2pt minus 2pt}

% ---- Hyperlinks ----
\usepackage[colorlinks=true,urlcolor=blue]{hyperref}	% For URL's. Automatically links internal references.

% ---- Code listings ----
\usepackage{listings} 					% Nice code layout and inclusion
\usepackage[usenames,dvipsnames]{xcolor}	% Colors (needs to be defined before using colors)

% Define custom colors for listings
\definecolor{listinggray}{gray}{0.98}		% Listings background color
\definecolor{rulegray}{gray}{0.7}			% Listings rule/frame color

% Style for Verilog
\lstdefinestyle{Verilog}{
	language=Verilog,					% Verilog
	backgroundcolor=\color{listinggray},	% light gray background
	rulecolor=\color{blue}, 			% blue frame lines
	frame=tb,							% lines above & below
	linewidth=\columnwidth, 			% set line width
	basicstyle=\small\ttfamily,	% basic font style that is used for the code	
	breaklines=true, 					% allow breaking across columns/pages
	tabsize=3,							% set tab size
	commentstyle=\color{gray},	% comments in italic 
	stringstyle=\upshape,				% strings are printed in normal font
	showspaces=false,					% don't underscore spaces
}

% How to use: \Verilog[listing_options]{file}
\newcommand{\Verilog}[2][]{%
	\lstinputlisting[style=Verilog,#1]{#2}
}




%======================================================
%=========== Body  ====================================
\begin{document}

\title{ELC 2137 Lab 1: Git and LaTeX Intro}
\author{Ivan Rios}

\maketitle


\section*{Summary}

The fundamental building blocks for Git And LaTeX were introduced and put into practice. First, LaTeX formatting principles were applied on the TeXstudio program, including manual cropping and table formatting along with basic font and structural techniques. Then, changes were uploaded to GitHub through git commits and git pushes in order to document and share the files.


\section*{Q\&A}

1. Rios2

2. An Itemize environment

3.  $y(t) =\frac{1}{2}e^t$

4. F6

\section*{Results}

\begin{figure}[ht]\centering
	\begin{tabular}{r|r|r}
		Binary & Hex & Decimal \\
		\midrule
		0000 & 0 & 0 \\
		0010 & 2 & 2 \\
	
		0100 & 4 & 4 \\
		0110 & 6 & 6 \\
		
		1000 & 8 & 8 \\
		1010 & A & 10 \\
		
		
		\bottomrule
	\end{tabular}\medskip



\includegraphics[scale=.5,trim={530 450 30 110},clip]{lab1_example_screenshot}
\caption{Example of a table and simulation waveform}
\end{figure}







\section*{Code}
\Verilog[caption=File-included Verilog code example,label=code:file_ex]{lab1_example_code.sv}




\end{document}
